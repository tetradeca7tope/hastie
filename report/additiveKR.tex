
\section{Problem Set up \& Algorithm}
\label{sec:additiveKR}

In this section we describe the problem, the proposed algorithm and its objective and our
methods for optimising the objective. 
% We begin with a brief review of Kernel Ridge Regression.

\subsection{Problem Statement \& Notation}
\label{sec:setup}

Let $\func : \Xcal \rightarrow \RR$ be the function of interest. 
Here $x = [x_1, \dots, x_D] \in \RR^D$ and $\Xcal \subset \RR^D$.
We have data $\XYn$ and wish to obtain an estimate
$\funchat$ of $\func$.
In this work, we seek additive approximations to the
function. That is, $\funchat$ can be expressed as,
\begin{align*}
\funchat(x) = \funchatii{1}(\xii{1}) + \funchatii{2}(\xii{2}) + \dots +
\funchatii{M}(\xii{M}),
\numberthis
\label{eqn:addAssumption}
\end{align*}
where $\xii{j} \in \Xcalj \subset \RR^{d_j}$ and $\funchatj:\Xcalj \rightarrow
\RR$. We shall refer to the $\Xcalj$'s as groups and the collection of all
groups $\bigcup_j \Xcalj$ as the decomposition.
We are particularly  interested in the case
where $D$ is very large and the group dimensionality is bounded-- i.e. $d_j \leq
d \ll D$. 

The work in \citet{hastie90gam} treats $\funchat$ as a sum of one
dimensional components. The decomposition here corresponds to
$\xii{j} = x_j$, $d_j = d =1\; \forall j$ and $M = D$. 
In this project, we would like to be more expressive than this model. We will
consider decompositions for which $d > 1$ and more importantly allows for
overlap between the groups.
\citet{ravikumar09spam} treat $\funchat$ as a sparse combination of one
dimensional functions. While this is seemingly restrictive than
\citep{hastie90gam}, the sparse approximation may provide favourable
bias-variance tradeoffs in high dimensions. Drawing inspiration from this, we
will consider models where $M$ is very large and seek a sparse collection of
groups to approximate the function - i.e. $\funchatj = \zero$ for several $j$.


\subsection{Kernel Ridge Regression}
\label{sec:krReview}

One method to formulate a regression problem is to consider the following
optimisation problem.


\subsection{Additive Kernel Ridge Regression}
\label{sec:addKR}

