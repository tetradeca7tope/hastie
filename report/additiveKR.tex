
\section{Problem Set up \& Algorithm}
\label{sec:additiveKR}

% In this section we describe the problem, the proposed algorithm and its objective and our
% methods for optimising the objective. 
% % We begin with a brief review of Kernel Ridge Regression.

\subsection{Problem Statement \& Notation}
\label{sec:setup}

Let $\func : \Xcal \rightarrow \RR$ be the function of interest. 
Here $\Xcal \ni x = [x_1, \dots, x_D] \in \RR^D$ and $\Xcal \subset \RR^D$.
We have data $\XYn$ and wish to obtain an estimate
$\funchat$ of $\func$.
In this work, we seek an additive approximation to the
function. That is, $\funchat$ can be expressed as,
\begin{align*}
\funchat(x) = \funchatii{1}(\xii{1}) + \funchatii{2}(\xii{2}) + \dots +
\funchatii{M}(\xii{M}),
\numberthis
\label{eqn:addAssumption}
\end{align*}
where $\xii{j} \in \Xcalj \subset \RR^{d_j}$ and $\funchatj:\Xcalj \rightarrow
\RR$. We shall refer to the $\Xcalj$'s as \emph{groups} and the collection of all
groups $\bigcup_{j=1}^M \Xcalj$ as the \emph{decomposition}.
We are particularly  interested in the case
where $D$ is very large and the group dimensionality is bounded-- i.e. $d_j \leq
d \ll D$. 

The work in \citet{hastie90gam} treats $\funchat$ as a sum of one
dimensional components. The decomposition here corresponds to
$\xii{j} = x_j$, $d_j = d =1\; \forall j$ and $M = D$. 
In this project, we would like to be more expressive than this model. We will
consider decompositions for which $d > 1$ and more importantly allows for
overlap between the groups. For e.g. $\funchat(x_1, x_2,
x_3) = \funchatii{1}(x_1) + \funchatii{2}(x_1, x_2) + \funchatii{3}(x_2, x_3)$.
\citet{ravikumar09spam} treat $\funchat$ as a sparse combination of one
dimensional functions. While this is seemingly restrictive than
\citep{hastie90gam}, the sparse approximation may provide favourable
bias-variance tradeoffs in high dimensions. Drawing inspiration from this, we
will consider models where $M$ is very large and seek a sparse collection of
groups to approximate the function - i.e. $\funchatj = \zero$ for several $j$.



\subsection{Additive Kernel Ridge Regression}
\label{sec:addKR}

We begin with a brief review on Kernel Ridge Regression.
One of several ways to formulate a nonparametric regression problem is to 
minimise an
objective of the form $J(f) = \sum_{i=1}^n \ell(f(X_i), Y_i) + \lambda P(f)$ 
over a nonparametric class of functions $\Fcal$.
Here $\ell$ is a loss function and $P$ is a term that penalises the complexity
of the function $f$. Several nonparametric regression problems such as smoothing
splines, natural splines and Kernel Ridge Regression can be written this way.
Central to \krrs is a positive semidefinite kernel 
$\kernel: \Xcal \times \Xcal \rightarrow \RR$ \citep{scholkopf01kernels}.
Then, $\Fcal$ is taken to be the reproducing kernel Hilbert
space (RKHS) $\Hcal_\kernel$ corresponding to $\kernel$, $P$ to be the squared
RKHS norm of $f$ and $\ell$ the squared error loss.  Accordingly, \krrs is
characterised via,
\[
\funchat = \argmin_{f \in \Hcal_\kernel} \sum_{i=1}^n (Y_i - f(X_i))^2 +
\lambda \|f\|^2_\Hcalk
\]
However, as mentioned previously  \krrs suffers from the curse of
dimensionality. To obtain an additive approximation using a given decomposition
$\bigcup_j \Xcalj$, we consider kernels $\kernelj :\Xcalj \times \Xcalj
\rightarrow \RR$ acting on each group and their associated RKHSs $\Hcalkj$.
Further, since we will set $M$ to be large and seek a sparse collection of
functions in our additive model we introduce an additional penalty term for
nonzero $\funcj$.
Putting it all together, our additive Kernel Ridge Regressin (\addkrr) is
characterised via the following problem where we jointly optimise over
$\funcii{1}, \dots, \funcii{M}$,
\begin{align*}
&\left( \funcii{1}, \funcii{2}, \dots, \funcii{M} \right) \;=\; 
\argmin_{\funcj \in \Hcalkj, j = 1,\dots, M} \;\;
  F\left( \{\funcj\}_{j=1}^M \right) \hspace{0.2in} \textrm{where, } \\
& F\left( \{\funcj\}_{j=1}^M \right)  \;= \;
  \frac{1}{2}\sum_{i=1}^n \Big(Y_i - \sum_{j=1}^M \funcj (\xj) \Big)^2 
  + \frac{\lambda_1}{2} \sum_{j=1}^M \|\funcj\|^2_\Hcalkj 
  + \lambda_2 \sum_{j=1}^M \indfone(\funcj \neq \zero)
  \numberthis \label{eqn:rkhsObjective}
\end{align*}
Our estimate for $\func$ is then $\funchat(\cdot) = \sum_j \funchatj(\cdot)$.

Via an argument that uses the representer theorem it is straightforward to show
that $\funcj$ will be in the linear span of the reprodcing kernel maps of the
training points $\Xjn$ -- i.e. $\funcj(\cdot) = 
\sum_j \alphaj_i \kernelj(\cdot, \Xj_i) $.
Then, the $j$\superscript{th} term inside the summations of the second and third
terms of equation~\eqref{eqn:rkhsObjective} can be written as 
${\alphaj}^\top \KKj \alphaj$ and $\indfone(\alphaj \neq 0)$.
Here $\KKj \in \RR^{n\times n}\; \forall
j$ such that $\KKj_{rc} = \kernelj(X_r, X_c)$.
Since the latter term is nonconvex we relax it via a group lasso type penalty.
After simplifying the first term, our optimisation objective can be written as,
$\aalpha = \argmin_{\aalpha \in \RR^{nM}} F(\aalpha)$ where,
\begin{equation}
F(\aalpha) = \frac{1}{2}\|Y - \sum_{j=1}^m \KKj \alphaj\|_2^2 + 
  \frac{\lambda_1}{2} \sum_{j=1}^M {\alphaj}^\top \KKj \alphaj +
  \lambda_2 \sum_{j=1}^M \| \alphaj \|_2.
\label{eqn:optObjective}
\end{equation}
Here $\alphaj \in \RR^n \; \forall j$, $\aalpha = [{\alphaii{1}}^\top, \dots, 
{\alphaii{M}}^\top]^\top  \in\RR^{nM}$. Given the solution to the above, our
estimate is obtained via $\funchat(\cdot) = \sum_{j=1}^M \sum_{i=1}^n \alphaj_i
\kernelj(\cdot, \Xj_i)$.
Equation~\eqref{eqn:optObjective} will the (convex) optimisation problem in our
algorithm.

\subsection{Practical Considerations}

All that is left to do to complete the specification of our algorithm is to
describe the kernel used. As is the case in several kernel methods, the choice
of the kernel is crucial for good empirical performance.
For each $\kernelj$ we use an RBF kernel, with scale parameter $\sigma_j$
and bandwidth parameter $h_j$.
\[
\kernelj(\xj_s, \xj_t) = \sigma_j \exp \left( \frac{\|\xj_s -\xj_t\|_2^2 } 
  {2h_j^2} \right)
\]
Here $\sigma_j$ captures the variation in the output and $h_j$ captures the
variation in the input.
In \krrs , these parameters along with the penalty coefficient $\lambda$ are
chosen via cross validation.
However, this is infeasible in our setting as $M$ is potentially very large. 
In our case we set $h_j = 1.5 \| \textrm{std}(\Xjn) \|_2 n^{\frac{-1}{4+d_j}}$.
Here $\textrm{std}(\Xjn)$ is the vector of standard deviations of the training
dataset on the coordinates belonging to group $j$. This choice of bandwidth is
motivated by the Silverman bandwidth and several kernel methods which choose 
a bandidth on the order $O(n^{\frac{-1}{2\beta+d_j}})$  where $\beta$ is the
smoothness of the function. We set $\sigma_j = \textrm{std}(Y) / \sqrt{M}$.
Finally, the penalty coefficients $\lambda_1$ and $\lambda_2$ are chosen via cross
validation.

