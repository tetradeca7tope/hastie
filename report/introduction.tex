

\section{Introduction}

Regression in high dimensions is an inherently difficult problem with known
lower bounds depending exponentially in dimension
\citep{gyorfi02distributionfree}. In this
project we intend to make progress in this problem by treating the function as
an additive model of lower dimensional components.
Using additive models is fairly standard in high dimensional regression
literature \cite{hastie90gam,ravikumar09spam,lafferty05rodeo}. However, in this
work we wish to consider additive models which are more general/ expressive than
previous work.

There are a number of potential nonparametric methods for modeling the low-order interaction terms.
One option is to use multidimensional splines. 
Thin plate splines \citep{thin-plate-splines:2003} extend 
spline-based nonparametric regression to multiple covariates,
although computational complexity increases dramatically with the order of interaction.
Natural thin plate splines extend one-dimensional smoothing splines.
These can be fit via penalized regression, and
complexity can be reduced by choosing knots on a grid rather than at each data point.
Thin plate (penalized) regression splines can alternatively be used;
this approach requires truncated SVD, but avoids choice of knots.
Tensor product splines \citep{tensor-product-splines:1994} can be used to construct 
multivariate splines as tensor products of single-dimensional splines.
The number of basis functions grows exponential with the interaction order, 
but they can be fit via penalized regression.


Another option is to model the low-order interaction basis functions implicitly using kernels.
There is existing research on using linear combinations of kernels for kernel learning,
called multiple kernel learning \citep{mkl-review:2011}.
Computationally efficient use of additive kernels has also been explored 
in Bayesian settings \citep{duvenaud11additivegps}.

Our work extends Sparse Additive Models (SpAM) \citep{ravikumar09spam} 
to multi-dimensional nonparametric basis functions.
For Sparse Additive Models, parameters are typically optimized using the backfitting algorithm.
Our work also extends recent work on 
Generalized Additive Models plus Interactions \citep{intelligible:2013}.
However, in this work the interaction model was assumed to follow a specific functional form,
leading to an optimization method tailored to their interaction model.
