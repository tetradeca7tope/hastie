\documentclass[10pt]{article}

% Packages
%%%%%%%%%%%%%%%%%%%%%%%%%%%%%%%%%%%%%%%%%%%%%%%%%%%%%%%%%%%%%%%%%%%%%%%%%%%%%%%%
\usepackage{calc}
\usepackage{amsmath, amssymb, amsthm}
\usepackage[margin=1in]{geometry}
\usepackage{parskip}
\usepackage{color,hyperref}
\usepackage{epsfig}
\usepackage{subfigure} 
\usepackage{verbatim}
% For algorithms
\usepackage{algorithm}
\usepackage{algorithmic}
% Add my definitions file
\usepackage{kky}
\usepackage{hastieDefns}

%%%%%%%%%%%%%%%%%%%%%%%%%%%%%%%%%%%%%%%%%%%%%%%%%%%%%%%%%%%%%%%%%%%%%%%%%%%%%%%%
% New Commands

%%%%%%%%%%%%%%%%%%%%%%%%%%%%%%%%%%%%%%%%%%%%%%%%%%%%%%%%%%%%%%%%%%%%%%%%%%%%%%%%
\newcommand{\toworkon}[1]{\textcolor{magenta}{[#1]}}
% \newcommand{\toworkon}[1]{}

% Title
%%%%%%%%%%%%%%%%%%%%%%%%%%%%%%%%%%%%%%%%%%%%%%%%%%%%%%%%%%%%%%%%%%%%%%%%%%%%%%%%
\makeatletter% since there's an at-sign (@) in the command name
\renewcommand{\@maketitle}{
  \parindent=0pt
  {\Large \bfseries {\@title}}  \\[1mm]
   \textbf{\@author} \hfill   \textit{\@date } \\[-1mm]
  \Hrule
	\vspace{0.4in}
}
\makeatother % resets the meaning of the at-sign (@)
\title{\textbf{Notation \& Setup}}
\author{}
\date{\today}

% Document
%%%%%%%%%%%%%%%%%%%%%%%%%%%%%%%%%%%%%%%%%%%%%%%%%%%%%%%%%%%%%%%%%%%%%%%%%%%%%%%%
\begin{document}
\maketitle

Hey, Let's use the following notation in our writeups and code ?

The first primal objective is,
\[
F_1(\aalpha) = \frac{1}{2n} \|Y - \sum_{j=1}^M \KKj \alphaj \|_2^2 
  + \frac{\lambda}{M} \sum_{j=1}^M \sqrt{\alphaj^\top \KKj \alphaj}
\]

Let the Cholesky decomposition of $\KKj$ be, $\KKj = \LLj \LLj^\top$ where
$\LLj$ is lower triangular\footnote{Matlab's \texttt{chol} gives the upper triangular
matrix, but my wrapper \texttt{stableCholesky} takes care of it. It also,
takes care of numerical issues in PSD matrices but be careful when using a
non-PSD matrix with it.}. Let  $\beta_j =  \LLjt \alphaj$. Then, we can write
the above objective as,
\[
F_2(\bbeta) = \frac{1}{2n}\|Y - \sum_{j=1}^M \LLj \betaj\|_2^2 +
  \frac{\lambda}{M}\sum_{j=1}^M \|\betaj\|_2
\]
Lets call this the second primal objective.


\section*{Coding}

Lets use the following standards, its basically what we are using right now.
\begin{itemize}
\item Maintain $\aalpha$ as an $\RR^{n\times M}$ matrix. Each column corresponds
to an $\alphaj$.
\item Maintain all the Kernel matrices in a $\RR^{n\times n\times M}$ tensor
\texttt{K}, where \texttt{K(:,:,j)} is $\KKj$.
\item Same for all the $\LLj$'s.
\item Lets use $n$ to denote the number of data points and upper case $M$ to
denote the number of groups -- even in the code.
\end{itemize}



% Bibliography
%%%%%%%%%%%%%%%%%%%%%%%%%%%%%%%%%%%%%%%%%%%%%%%%%%%%%%%%%%%%%%%%%%%%%%%%%%%%%%%%
\bibliographystyle{alpha}
\bibliography{kky}


% Appendix
%%%%%%%%%%%%%%%%%%%%%%%%%%%%%%%%%%%%%%%%%%%%%%%%%%%%%%%%%%%%%%%%%%%%%%%%%%%%%%%%
% \appendix


\end{document}

